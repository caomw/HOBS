\section{Discussion}
\label{sec:discussion}

\ben{cannot focus on writing this part clearly at 4am... I am just laying down whatever has popedup my heads and plan to refine it tomorrow.}

\subsection{Generalization of our results}
\label{sec:gener-our-results}
This paper describes three iterative designs and either of them can be applied to a more broader range. First, the scan + refinement model is generic. Second, with a combination of signal reception and strength measurement, together with the signal strength model, a system can usually do better. Third, multiple types of sensors combined can overcome a lot of issues. 

\subsection{Using Google Glass}
\label{sec:using-google-glass}
In our current implementation, we have chosen to use Google Glass as the main head-worn computing device. This decision was made primarily because Glass has the display, touchpad, and IMU sensors and programming Glass is easy (standard Android). But this doesn't exclude other head-worn devices. 

Comments about that our system made the assumption that users have to wear Google Glass first doesn't hold. The work is more about the design exploration of how head orientation can be used for targeting. 

\subsection{Gaze Tracking}
\label{sec:gaze-tracking}
The reason I brought up it here is because gaze tracking seems to be mentioned by multiple people and having a discussion here is good to clear the questions.




%%% Local Variables: 
%%% mode: latex
%%% TeX-master: "uist14"
%%% End: 
