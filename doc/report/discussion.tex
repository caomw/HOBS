
\section{Discussion}
\label{sec:discussion}

% Discuss the problems we have encountered and other random stuffs in this section
% In this section, we will touch something that is related to our project, or pops up during our thinking that might be beneficial to people who consider working on similar topics. We will first talk about the attention-based interaction and its limitation. Then specific to our project implementation, we would like to express some of our experience.

\textbf{IR sending and receiving:} We found that it is an intuitive and fairly accurate method for detecting the user's attention. However, there are two limitations under current design: (1) We want the user be able to control the appliances as long as they can see it. However, in some cases where the user is blocked by other objects or in a particular angle, being able to see part of the appliance does not necessarily mean the user can see the IR receiver. (2) When the wall is somewhat reflective or has some reflective materials such as white board hanging on it, the IR might bounce back and trigger the appliances in back of the user. We are discussing ways to mitigate the problem.

\textbf{XBee:} The communication between XBees are complicated. For one, the signals are transmitted one by one instead of in a packet. Hence, we need to handle with extra care. In addition, sometimes multiple clients can be sending signals simultaneously. We added a delay to each client that is proportional to their client IDs to avoid conflicts.

\textbf{Slider:} The slider we ordered is actually a soft potentiometer that returns a value between 0 to 1023 when pressure is applied to it. And the returned value reflects the position being pressed along the x-axis. To detect tap is easy. To detect hold, we need to handle accidentally release since the soft potentiometer requires certain amount of strength and the glass could move when the user is sliding along the temple. We also added a bounce handling so that the main program is only notified of slide change if a certain threshold of value is changed on the slider. Finally, we added the double tap gesture and it was a bit tricky to tell it from accidentally release. We believe that we could use a better piece
%purchase a better piece 
in our next prototype so that we can concentrate more on the interaction design rather than the basic components.

% \subsection{Interaction in Attention}
% \label{sec:inter-attent}

% During our brainstorming and discussion of this idea, we have identified some tasks where user might not pay their attention to the device they are controlling. One such example is giving presentation. The attention of presenter is to interact with the audience, but this is achieved through controlling the slides and projection.

% {\color{red} a very short brainstorming would be able to finish this part.}


% \subsection{Experience learned in this project}
% \label{sec:exper-learn-this}

% XBee setting up. 

%%% Local Variables: 
%%% mode: latex
%%% TeX-master: "main"
%%% End: 