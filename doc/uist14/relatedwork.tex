%!TEX root = uist14.tex
\section{Background and Related Work}
Our approach is related to head and eye-controlled interfaces, selection techniques, and prior work on hardware devices for pointing in physical spaces.
\subsection{Head and Eye Input}
Head movement has long been used for virtual camera control in VR applications~\cite{pausch_user_1993} and as an assitive input technology for cursor control of desktop applications~\cite{radwin1990method}. However,
Card, summarizing other work, shows that neck muscles are a poor muscle group for pointing in general: neck muscles only have a bandwidth of about $4.2bits/s$ (compared to $23bits/s$ of wrist muscles used by a standard mouse~\cite{Card:1991:MAD:123078.128726}. This prior work often focused on used head orientation for control of graphical interfaces; in contrast, we apply this modality to selection in physical spaces, where users can also move around.

Researchers have also developed methods for gaze control of existing graphical user intefaces~\cite{kumar2007eyepoint}. While there are wearable gaze trackers~\cite{bulling2009wearable}, turning information about a concrete point in space where a user is looking into a selection requires either a map of each space with known location and orientation of the user. Our system works through point-to-point IR communication and does not require any a priori knowledge of where objects are located. An alternative would be to equip all objects of interest in the environment with cameras that watch the user~\cite{smith2013gaze}, though such an approach would be computationally expensive and may not work at greater distances.
%Our work is closer in spirit to Selker's headworn system~\cite{Selker:2001:EGE:634067.634176}.

\subsection{Area Cursors}
A central hypothesis of this paper is that the area selection paradigm is well matched to head orientation input. In area selection cursors in graphical user interfaces, users do X and then Y.
\bjoern{Need a discussion of area cursors~\cite{kabbash1995prince,worden1997making,Findlater-uist2010}.}

\subsection{Pointing in physical spaces}
\bjoern{take this from prior CHI submission.}

\bjoern{TODO: Add discussion of ~\cite{swindells_that_2002,merrill_augmenting_2007}.}
