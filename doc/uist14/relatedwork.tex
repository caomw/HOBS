%!TEX root = uist14.tex
\section{Background and Related Work}
Our approach is related to head- and eye-controlled interfaces, area cursors and prior work on hardware devices for pointing in physical spaces.

\subsection{Head and Gaze Input}
Head movement has long been used for virtual camera control in VR applications~\cite{pausch_user_1993} and as an assistive input technology for cursor control of desktop applications~\cite{radwin1990method}. However, human neck muscles have a much lower bandwidth than other muscle groups, e.g., the wrist~\cite{card_morphological_1991}.

%Card, summarizing other work, shows that neck muscles are a poor muscle group for pointing in general: neck muscles only have a bandwidth of about $4.2bits/s$ (compared to $23bits/s$ of wrist muscles used by a standard mouse~\cite{Card:1991:MAD:123078.128726}. 

Prior work often focused on head orientation for controlling graphical interfaces; in contrast, we apply this modality to selection in physical spaces.

Gaze can also be used to control graphical user interfaces~\cite{kumar2007eyepoint}. While there are wearable gaze trackers~\cite{bulling2009wearable}, turning information about a concrete point in space where a user is looking into a selection requires a map of each space with known target locations. Our system works through point-to-point IR communication and does not require an {\em a priori} map. Target objects in the environment can also be equipped with individual cameras that watch the user~\cite{smith2013gaze,vertegaal2005media}. Such an approach can enable similar benefits as our approach, but is computationally more expensive than our single-pixel sensor solution and may not work at greater distances or angles, because it relies on finding the user's pupils in a camera image.

%Our work is closer in spirit to Selker's headworn system~\cite{Selker:2001:EGE:634067.634176}.

\subsection{Area Cursors}
In GUI area cursors, the activation area of the cursor is enlarged, which facilitates acquiring smaller targets~\cite{kabbash1995prince}. Area cursors are especially appropriate for individuals with motor control impairments or difficulties~\cite{worden1997making,findlater2010enhanced}. We argue that head orientation pointing has similar challenges (limited pointing performance and accuracy). In all area cursors, the bigger cursor activation area can lead to multiple targets being selected and disambiguation is needed. 

We conceptualize area pointing as a two-stage process: in the {\em coarse} phase, which we call {\em scanning}, users move so the activation area intersects with the target object (and possibly other, unintended targets). In the {\em refinement} phase, they adjust so only the intended target will be selected. Many disambiguation techniques are possible for refinement -- this paper describes the trade-offs between several of them.

\subsection{Pointing in physical spaces}
Rukzio studied alternative methods for selecting devices in physical spaces and found that users strongly preferred either tapping target appliances with a mobile device or pointing at a distance to browsing a list~\cite{rukzio_experimental_2006}.

Several approaches to spatial selection with handheld devices exist for controlling appliances~\cite{beigl_point_1999,patel_2-way_2003,wilson_xwand:_2003,schmidt_picontrol:_2012,kemp_point-and-click_2008} and exchanging information with smart infrastructure sensor networks ~\cite{lifton_tricorder:_2007,mittal_ubicorder:_2011,costanza_sensortune:_2010}. In some techniques, users select objects of interest with laser pointers. The laser dot provides immediate visual feedback to the user about what is being selected; however, its small target area makes it poorly matched to head orientation input.

Other approaches rely on virtual room models in which a user's location is estimated using IMU-based orientation sensing~\cite{wilson_xwand:_2003,lifton_tricorder:_2007} -- in contrast, our technique does not require a static map ahead of time.

Handheld projectors can both display a user interface in space and communicate control information optically, e.g., by encoding information temporally (using Gray codes in Picontrol~\cite{schmidt_picontrol:_2012} and RFIG~\cite{raskar_rfig_2004}) or spatially (using QR codes in the infrared spectrum in SideBySide~\cite{willis_sidebyside:_2011}). Our solution is similar in spirit but relies on simple low-cost IR emitters and detectors. 
%However, visible tags at appropriate sizes may be rejected by users because of their negative aesthetic effect on the space.

Other targeting systems use IR with handheld pointers~\cite{swindells_that_2002} as well as wearable devices such as rings and Bluetooth audio earpieces~\cite{merrill_augmenting_2007} to connect to smart devices. Our system tackles an unresolved issue of such IR-based approaches -- navigating an area dense with potential targets.
