%!TEX root = sui14.tex
\section{Background and Related Work}
Our approach is related to head- and eye-controlled interfaces, area cursors and prior work on hardware devices for pointing in physical spaces.

\subsection{Head and Gaze Input}
Head movement has long been used for virtual camera control in VR applications~\cite{pausch_user_1993} and as an assistive input technology for cursor control of desktop applications~\cite{radwin1990method}. However, human neck muscles have a lower bandwidth than other muscle groups, e.g., the wrist~\cite{card_morphological_1991}.

%Card, summarizing other work, shows that neck muscles are a poor muscle group for pointing in general: neck muscles only have a bandwidth of about $4.2bits/s$ (compared to $23bits/s$ of wrist muscles used by a standard mouse~\cite{Card:1991:MAD:123078.128726}. 

Prior work often focused on head orientation for controlling graphical interfaces; in contrast, we apply this modality to selection in physical spaces.

Gaze can also be used to control graphical user interfaces~\cite{kumar2007eyepoint}. While there are wearable gaze trackers~\cite{bulling2009wearable}, turning information about a concrete point in space where a user is looking into a selection requires a map of each space with known target locations. Our system works through point-to-point IR communication and does not require an {\em a priori} map. Target objects in the environment can also be equipped with individual cameras that watch the user~\cite{smith2013gaze,vertegaal2005media}. Such an approach can enable similar benefits as our approach, but is computationally more expensive than our single-pixel sensor solution and may not work at greater distances or angles, because it relies on finding the user's pupils in a camera image.

%Our work is closer in spirit to Selker's headworn system~\cite{Selker:2001:EGE:634067.634176}.

\subsection{Area Cursors}
\changes{
In 2D area cursors for GUIs, the activation area of the cursor is enlarged, which facilitates acquiring smaller targets~\cite{kabbash1995prince}. We argue that head orientation pointing has analogous characteristics (limited pointing performance and accuracy). Area cursors are especially appropriate for individuals with motor control impairments or difficulties~\cite{worden1997making,findlater2010enhanced}. Similar ideas have also been extended into 3D to provide selection with progressive refinement in 3D scenes~
\cite{bacim2013design}. In all area and space cursors, the large activation area can lead to multiple targets being selected and disambiguation is needed. 

We conceptualize area pointing as a two-stage process: in the {\em coarse} phase, which we call {\em scanning}, users move so the activation area intersects with the target object (and possibly other, unintended targets). In the {\em refinement} phase, they adjust so only the intended target will be selected. Many disambiguation techniques are possible for refinement -- this paper describes the trade-offs between several of them.
}

\subsection{Pointing in Physical Spaces}
Rukzio studied alternative methods for selecting devices in physical spaces and found that users strongly preferred either tapping target appliances with a mobile device or pointing at a distance to browsing a list~\cite{rukzio_experimental_2006}.

Several approaches to spatial selection with handheld devices exist for controlling appliances~\cite{beigl_point_1999,patel_2-way_2003,wilson_xwand:_2003,schmidt_picontrol:_2012,kemp_point-and-click_2008} and exchanging information with smart infrastructure sensor networks ~\cite{lifton_tricorder:_2007,mittal_ubicorder:_2011,costanza_sensortune:_2010}. In some techniques, users select objects of interest with laser pointers. The laser dot provides immediate visual feedback to the user about what is being selected; however, its small target area makes it poorly matched to head orientation input.

Other approaches rely on virtual room models in which a user's location is estimated using IMU-based orientation sensing~\cite{wilson_xwand:_2003,lifton_tricorder:_2007} -- in contrast, our technique does not require a static map ahead of time.

Handheld projectors can both display a user interface in space and communicate control information optically, e.g., by encoding information temporally (using Gray codes in Picontrol~\cite{schmidt_picontrol:_2012} and RFIG Lamps~\cite{raskar_rfig_2004}) or spatially (using QR codes in the infrared spectrum in SideBySide~\cite{willis_sidebyside:_2011}). Our solution is similar in spirit but relies on simple low-cost IR emitters and detectors. \bjoern{need a stronger statement here}
%However, visible tags at appropriate sizes may be rejected by users because of their negative aesthetic effect on the space.

Other targeting systems use IR with handheld pointers~\cite{swindells_that_2002} as well as wearable devices such as rings and Bluetooth audio earpieces~\cite{merrill_augmenting_2007} to connect to smart devices. Our system tackles an unresolved issue of such IR-based approaches -- navigating an area dense with potential targets.

\subsection{Markers and Vision Methods}
\achal{Not sure where to put this, feel free to move.}
\ben{For CV, I think related work is one option to go; however, the requirement of being related work is that -- we need to reference them. There are a few examples ``require large, obtrusive tags, and inevitably lack feedback from passive devices in the environment'' which should be referenced. That's why I am thinking of putting this to discussion.}
\changes{Many alternative solutions for detecting devices in contained spaces
rely on passive tags on devices, combined with computer vision methods for
detection. Unfortunately, these methods either impose significant constraints
on the camera used for detection, or require large, obtrusive tags, and
inevitably lack feedback from passive devices in the environment. \cite{Bokode},
for example, relies on visually imperceptible codes, but has significant
drawbacks: despite using a 12.1 megapixel DSLR camera, detections are no
longer robust past an angle of 20. Similarly, in our experiments, we found
that striking a balance between the size of the tag, the prices of the
device, and the computational cost of detection was unfeasible with current
cameras and algorithms.}

\ben{Reviewers suggest considering computer vision-based techniques with markers in the environment (such as QR codes, or Bokode [Mohan] which extends the working distance to a few meters for visual tags). We considered this direction and found several drawbacks, including computational complexity, low accuracy when targets are far, the absence of real-time environment feedback with passive tags, and aesthetic issues with QR codes. While we don't claim our choice of IR is optimal, it has important advantages -- it is low-cost, readily available and quite suitable for area selection. We can weaken our claim and add explanations of how disambiguation techniques might generalize to other methods of implementing target identification.}

\achal{Read and discussed the Bokode paper and CV in general. Unsure about what
to do regarding "weakening our claim."}

%\ben{new related works from reviewers}. \bjoern{folded in}
%\changes{RFIG Lamps and photosensing wireless tags [Raskar et al] use a projector and data is encoded in each pixel so that tags can localize themselves for target selection. Our IR emitter, a relatively low-cost solution, doesn’t have this feature but shares the commonality of covering an area for the ease of selection. At the end of their paper, they envision an IR-based system to solve ambient light problems. Our work lies in that direction and also contributes evaluations with user studies.
%Progressive Refinement [Bacima] studies several progressive refinement selection modalities, which is close to our two-stage selection; but our contexts are head-orientation as it reflects users' point of interest.
%}
