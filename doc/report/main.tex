\documentclass{sig-alternate}

\usepackage{times}
\usepackage{graphics}

\usepackage{subfigure}
\usepackage{booktabs}
\usepackage{colortbl}
\usepackage{tabularx}
\usepackage{color}
\usepackage{xspace}
\usepackage{hyperref}    % Creates hyperlinks from ref/cite
\hypersetup{pdfstartview=FitH}
\usepackage{graphicx}    % For importing graphics
\usepackage{url}         %

\hypersetup{%
pdftitle={Project Glass 2.0}, pdfauthor={Ben Zhang, Sean Chen}, pdfkeywords={HCI}, bookmarksnumbered, pdfstartview={FitH}, colorlinks,
citecolor=black, filecolor=black, linkcolor=black, urlcolor=black,
breaklinks=true,}

\renewcommand{\arraystretch}{1.2} % Space out rows in tables

\newcommand{\ml}[1]{{\color{green} {\it ML: #1}}}

% No space between bibliography items:
\let\oldthebibliography=\thebibliography
  \let\endoldthebibliography=\endthebibliography
  \renewenvironment{thebibliography}[1]{%
    \begin{oldthebibliography}{#1}%
      \setlength{\parskip}{0ex}%
      \setlength{\itemsep}{0ex}%
  }%
  {%
    \end{oldthebibliography}%
  }

%\pagenumbering{arabic}  % Arabic page numbers for submission.  Remove this line to eliminate page numbers for the camera ready copy

\begin{document}

% use this command to override the default ACM copyright statement
% (e.g. for preprints). Remove for camera ready copy.
%\toappear{Submitted for review to IPSN 2012.}
% \conferenceinfo{ConfName} {Date, Location}
% \CopyrightYear{Year} 
% \crdata{978-1-4503-1227-1/12/04} 
% \clubpenalty=10000 
% \widowpenalty = 10000


% to make various LaTeX processors do the right thing with page size
\special{papersize=8.5in,11in}
\setlength{\paperheight}{11in}
\setlength{\paperwidth}{8.5in}
\setlength{\pdfpageheight}{\paperheight}
\setlength{\pdfpagewidth}{\paperwidth}

\title{Project Glass 2.0 -- Interaction with Physical Devices through Attention}

\author{
{Ben Zhang, Sean Chen}\\
\affaddr{University of California, Berkeley}\\
%\affaddr{}\\
\email{benzh@eecs.berkeley.edu, sean.yhc@berkeley.edu}
}

\maketitle

\begin{abstract}
With the increasing interest of assigning intelligence to everyday object, many approaches of interacting with them fall back to smartphone based applications. We argue about the inconvenience, overhead, and distraction in such an interaction paradigm, and propose a more natural solution. A glass-based controller which enables user to express their interest in interaction directly simplifies the process of selection in a smart-home or smart-office senario. We present our thoughts, design and evaluation of such a system in this paper to motivate researchers in considering ``interaction in attention''. In the meantime, we noticed that Google Glass might serve as the perfect enabling technology, though in our prototype we've adopted simple infrared approach for proof-of-concept.
\end{abstract}

%\category{C.2.1}{Computer-Communication Networks}{Network Architecture and Design}[Wireless communication]
%\keywords{Collection, CTP, Sensor Network, Routing}

% \category{B.0}{Hardware}{General}
% \category{B.4}{Hardware}{Input/Output \& Data Communication} 
% \category{H.4.m}{Information Systems Applications}{Miscellaneous}

% \terms{Design, Experimentation, Measurement} 
\keywords{HCI, Interaction, Attention, Infrared, Glass}

\section{Introduction}
\label{sec:introduction}

There has been the vision of smart-home and smart-office where everything around is intelligent enough to assist human in everyday life. The development of radio technology, sensors, and actuators are making this dream come true. You can buy those basic components online \cite{SmartHome, NinjaBlocks} and many hobbyists have already created their own smart environment (see \cite{BRAD} as an example). Recently, with the the prevalence of smartphones, many systems \cite{SmartThings, Lockitron} have been integrated into smartphone applications to lower the barrier for ordinary users.

Admittedly, people have been used to accessing the Internet, conversing with friends, and even dealing with online business through their smartphones. These handheld devices serve as an important portal for users to be connected to the Internet. And now many systems are making smartphones the portal to interact with physicalworld. Given all the benefits we can get by using smartphones -- networking,  display, gesture input, and increasing computability,  we argue that traditional GUI on mobile platform still imposes difficulty in interacting with physical objects. Take iPhone for example, users are required to browse a list of possible icons and then select the right one. Even on Android platform, you can create widget to simplify the browsing overhead. We hold the opinion that squeezing physically distributed objects onto a-few-inch screen is not the right approach. And a more direct approach is desirable.

In this project, we aim to explore the ability of direct interaction by using glass-based attention tracking to pin-point the target. 

% explain why we need glass form factor


The rest of this paper is structured as follows. In Sec.\,\ref{sec:related-work}, we will give an overview of some related projects. Then we present our approach in detail about the system design and implementation in Sec.\,\ref{sec:design} and Sec.\,\ref{sec:implementation} respectively. Then Sec.\,\ref{sec:evaluation} shows our evaluation results and we provide an in-depth discussion (Sec.\,\ref{sec:discussion}) about different kinds of interaction since not all are suitable for our system. Sec.\,\ref{sec:conclusion} concludes this paper.


%%% Local Variables: 
%%% mode: latex
%%% TeX-master: "main"
%%% End: 

\section{Related Work}
\label{sec:related-work}

In this section, we will summarize some prior work and how it relates to our project in three folds: Tangible User Interface (TUI), Glass-based device, and Attention Aware System (AAS), discussing each in turn. 

\subsection{Tangible User Interface}
\label{sec:tang-user-interf}

The concept of Tangible User Interface was proposed in \cite{Ishii:1997:TBT:258549.258715}, and according to the original paper, such system would enable user to ``grasp \& manipulate bits in the center of users' attention by coupling the bits with everyday physical objects and architectural surfaces''. Thereafter, numerous TUI systems (such as \cite{Nanayakkara:2012:EEF:2212776.2212382, Patel:2006:IPA:2094945.2094962}) are developed. \cite{Merrill:2007:ALP:1758156.1758158} is one that is closest to our system, which base users' interaction with physical objects on people's natural looking, pointing and reaching metaphor. However, these systems have limitations that they focus too much on sensing the physical world, while the actuation is mostly done in cyber world. Recent years, many startups \cite{SmartThings, NinjaBlocks, Lockitron} tries to ``hack'' the physical world to achieve the envisioned Internet of Things by enabling actuating devices from phone/web portal. 

Our system combines some merits from prior work. As has been pointed in Sec.\,\ref{sec:introduction}, we are also using people's natural looking as attention tracking as in \cite{Merrill:2007:ALP:1758156.1758158}, but we are not limited to browsing or searching physical objects. We provide such capability of interacting directly with physical devices. This also improves some of existing work that is based on smartphones or web portal \cite{SmartThings, NinjaBlocks}.

\subsection{Glass Form-factor}
\label{sec:glass-form-factor}

\cite{mann2004continuous}, as the earliest wearable device, uses the form factor of glass to serve as a logging machine that records user's’ daily life. According to one review paper \cite{morris2010emerging}, there are many glass-based device developed for the past decades. Those who use glasses as input have primarily focused on gaze-tracking \cite{Selker:2001:EGE:634067.634176, Nagamatsu:2010:MDG:1753846.1753983}, and the projects that uses glasses for available output device works hard to achieve virtual reality \cite{Lumus, GoogleGlass}.

The most recent and potentially most impactful project \cite{GoogleGlass} integrate many crucial components -- camera, projection, microphone, speaker, IMU and networking -- into a single glass that can achieve virtual reality to an unimaginable extent. We do believe this would become a powerful platform for future interaction development. However, so far the vision is still being limited to digital world. In contrast, our project explores the potential of using glasses for direct interaction with physical devices, rather than performing ``digital'' tasks.

\subsection{Attention Aware System}
\label{sec:attent-aware-syst}

Microsoft Research's project ``Attentional User Interface'' \cite{horvitz2003models} explores how attention can be employed to enhance human-computer interaction, from the lesson learned in human-human interaction. Attention detection can also assist interface design to avoid context shifting overhead, e.g. the peripheral displays and the notification level should be adjusted according to user's attention \cite{parkdesigning}. 

These work researched more on how the system should manage or adapt to users' attention. While for us, we want user to express their attention directly and use this to address the target for interaction.

%%% Local Variables: 
%%% mode: latex
%%% TeX-master: "main"
%%% End: 

\section{Design}
\label{sec:design}

% ok, I have to admit that when I start writing, I haven't figured out what to write.
% so kinda messy.

% this section is used to discuss the design considerations of the system
% starting from a motivating example, and then extract the key requireements, then about system design

In this section, we will talk about our thoughts on enabling direct interaction with physical device, driven by users' attention. We will first discuss one example application which motivates the work in this project. We also seek to summarize the key design requirements, and use them for the system design. Then several possible approaches are presented as an exploration o f the design space. To build a proof-of-concept system, we've simplified several tasks and implement a working system. The detailed implementation details will be covered in the next section.

\subsection{Example Application}
\label{sec:example-application}

Our work is primarily motivated by the observation that in CS294-84 course at Berkeley, the instructor has to go to the switch to turn off the light during presentation. The separation of device and their controller makes it possible for people control things that are not reachable (such as the lights on the ceiling). However, such indirection introduces additional overhead since people still need to locate the controller and then interact. Remote control might mitigate such problem, but again a controller for a device is not a scalable solution. Squeezing controllers into the smartphone display is also not the right approach. And these points will be more clear when we show our example application -- Smart Home.

% {\bf Smart Home}:
% Sean, maybe you can elaborate on this application more to motivate.
Imagine in a smart-home envinroment that most device are controllable remotely. When the users are watching TV shows, they want to turn off the lights and turn up the volume so they can enjoy this memorable time. Such ordinary-life tasks have been simplified so that they are able to issue ``turn off'' commands to the lights by simply staring at the light plus a specific gesture. Then they turn their head back to the TV, and again simply ``ask'' the TV to turn up the volume. When transitioning, the disconnection with lights and connection with TV happens seamlessly, with the correct feedback. Of course all the magic has to be resolved by an always-available device, such as glasses, rings, watches, or any other proper form factor. We will hold the discussion to the design space exploration. 

Occasionally, the user needs to perform some advanced configuration to the TV, such as adjusting the white balance, or the brightness. Since the need of such complicated tasks are not frequent, and it will be less efficient to embed these features into the always-available device (which are designed to be small and easy-to-access, rather than too flexible). The smart-home falls back to a more generic approach which relies on touch screen or other rich input device. In this example case, since the user's attention is still on TV, his smartphone pops up the control menu for advanced usage automatically. Such menu could also just pop up on walls or other large interactive devices \cite{unPad:eWallpaper, MSVision}, but the attention-based interaction simplifies the selection on those flexible devices.

\subsection{System Requirements}
\label{sec:system-requirements}

From what we have described in the previous section, to achieve such direct interaction in attention, we have identified several key requirements.

% leads to the adoption of glass
The first and foremost is the ability of always-available sensing and acutation. Since our target is to simplify everyday task in smart-home/office scenario. Any cubersome solution is not desired.

Secondly, users are directly interacting with physical device. Since a screen is not always available, there should be some feedback indicator to explicit tell users what is going on.

As discussed before, we envision the simple always-available device would not be able to cover all tasks, especially some advanced ones. We will have to design the system to enable falling back.

For any human-centric system, additional requirements like responsiveness would aslo be required. 

\subsection{System Design}
\label{sec:system-design}

In this section, we will present several design consideration that fulfills the requirements. From \cite{Merrill:2007:ALP:1758156.1758158}, they tried the solution of putting IR into earphones, rings to enable pointing metaphor. Other than that direction, recently published Google Glass project motivates us to consider glasses form factor. Glasses are natural in align with human's vision, and can reflect users' attention in a more precise way. But given the form factor of glasses, what might be a good approach to achieve sensing and actuation remains a challenge for us.

% kinda feel lost in describing

In our discussion, we've come up with three possible approaches. The first is a purely glass solution. The second is augmented by some computer vision detection. The third  will be a combination of glasses and smart watch. 

% Lessons learned from \cite{Bellotti:2002:MSS:503376.503450} have also guided us in thinking the Five A's when designing novel interaction system. 


%%% Local Variables: 
%%% mode: latex
%%% TeX-master: "main"
%%% End: 



\section{Implementation}
\label{sec:implementation}

In this section, we details our implementation on Arduino \cite{Arduino} platform. The whole system is comprised of a glass module, numbers of clients, and a gateway. We will start by the clients, which responds to different message sent by others. The complexity of coordination is pushed to the glass, and only minimal function are provided in the gateway.

\subsection{Clients}
The clients are comprised of a main Arduino board, an IR receiver, an XBee radio, and various actuators. In our current system, we have a relay which can control the AC power plug, and we have a USB connected computer to control video playing. 

The main function of client is to respond IR signal, and communicate with glass to receive corresponding commands. When there is no glass initiating connection, the client simply lives in {\it IDLE} mode; while when they receive IR signal, indicating the user is expressing interest in interacting with it, the client responds with an XBee acknowledgement and goes into {\it PENDING} state. In this case, since the client doesn't know how many other clients have also responded to the glass, it will wait until a connection signal. If there are multiple clients waiting to be verified, the glass will coordinate all of them by sending the right verifying message. The active client that is being verified will blink faster, as visual feedback to users. When it receives the connection message, it goes to {\it CONNECTED} state and is ready to take commands. To summarize the behavior, we have a state machine in Fig.\,\ref{fig:clientFSM} for illustration. 

\begin{figure}
  \centering
  \includegraphics[width=\linewidth]{../figs/clientFSM.pdf}
  \caption{FSM model for client. For each transition, ``e'' stands for event, and ``a'' stands for action}
  \label{fig:clientFSM}
\end{figure}

Within {\it CONNECTED} state, the clients' behaviors are different from each other according to the device it is connected to. In our prototype, we have two different types of devices. The first is a relay, which can then control the whole AC power supply. To control the relay, we only need to set a digital pin to {\it LOW} or {\it HIGH}, and each action is encoded in the command set from the glass. The second is a computer that is used to play video. And the direct interaction includes adjusting the volume and pause/play. These commands are sent to the computer through USB serial data.

\subsection{Glass}
\label{sec:glass}

The Glass is complicated in two folds. First, we move the complexity of coordinating all clients to the Glass. Second, we have to handle user gesture on Glass. For a clean design, we separate the gesture detection and XBee transmission into two customized library, and only expose interesting information to the main program.

The whole work flow is as follows. When the user hasn't shown any interest of interaction, the system lives in {\it IDLE} state. When ``tap'' is detected, the Glass will broadcast IR signal to inform the devices which is in front of the user (the one that can receive IR signal). Then the Glass waits for a specific period of time for acknowledge message sent by clients. The timer in {\it WAIT} state now is set to 1 second. When timer expires, Glass will go to {\it IDLE} if no client has responded, or {\it CONFIRM} if it receives just one message. However, the complicated case is when there are multiple clients that have received IR signal and responded. Though we expect such scenario is rare, there is still chance that the physical devices are put together and it is necessary to differentiate them. In this case, the system goes to {\it VERIFY} state, and send out corresponding verifying message based on user's gesture. Once the user confirmed the selection, it goes to {\it CONNECTED} state where it's ready to send out commands. Again, to illustrate the flows, we have the FSM in Fig.\,\ref{fig:glassFSM}.

\begin{figure}
  \centering
  \includegraphics[width=\linewidth]{../figs/glassFSM.pdf}
  \caption{FSM model for Glass. For each transition, ``e'' stands for event, and ``a'' stands for action}
  \label{fig:glassFSM}
\end{figure}

\subsection{Gateway}
\label{sec:gateway}

The gateway provides access to the personal area network for computers, and presumably this can be further open to the Internet. Since there have already been many products \cite{NinjaBlocks, Lockitron} that essentially function in this way, we do not spend much time on this aspect in our project. 

Just for completeness, we wrote a python script that controls a USB XBee adaptor, so that we can remotely control those devices without the Glass and IR initiation.

%%% Local Variables: 
%%% mode: latex
%%% TeX-master: "main"
%%% End: 

\section{Evaluation}
\label{sec:evaluation}

\ben{still thinking how to proceed.}
In this section, we evaluate our targeting system by benchmarking its performance with an user study. We first describe our design of the physical targeting study, followed by the results in different scenarios.

\subsection{Apparatus}
We deployed 10 wireless nodes in an indoor environment at various distance and density. Then the participant is asked to stand in a fixed position in the room and look down before the targeting request is made. We randomly generate a target and highlight it using a yellow LED, and then measure the amount of the time for a user to achieve a targeting transaction. 

\subsection{Methodology}
\label{sec:methodology}

We break down the time for a complete target acquisition to the following pieces:
$t_{total}=t_{locate}+t_{reorient}+t_{disambiguate}+t_{tap}$
And during the study, we video-taped the participants' behavior and measure each pieces of the time. The experiment setup is similar to a Fitts' Law target acquisition task. We highlight one of the targets and ask the user to make the selection.

In our previuos studies, we have made a comparison of using IR and Glass List UI. There we reach an conclusion that once the targets numbers have exceeded a certain value (6 in a single room), then list selection (using Google Glass List UI) will be worse than IR targeting. However, the average results there comes from cases where disambiguation is not needed and needed. And disambiguation is a time-consuming part. So in this paper we focus more study on the disambiguation techniques.


\subsection{IR intensity based}
We evaluate how much gain we can get by performing an automatated disambiguation if the IR intensity reading is available. We compare against the case where no IR intensity is available. 

\subsection{IR intensity together with Google Glass}

\subsection{Manual disambiguation}



%%% Local Variables: 
%%% mode: latex
%%% TeX-master: "uist14"
%%% End: 

\section{Discussion}
\label{sec:discussion}

\ben{cannot focus on writing this part clearly at 4am... I am just laying down whatever has popedup my heads and plan to refine it tomorrow.}

\subsection{Generalization of our results}
\label{sec:gener-our-results}
This paper describes three iterative designs and either of them can be applied to a more broader range. First, the scan + refinement model is generic. Second, with a combination of signal reception and strength measurement, together with the signal strength model, a system can usually do better. Third, multiple types of sensors combined can overcome a lot of issues. 

\subsection{Using Google Glass}
\label{sec:using-google-glass}
In our current implementation, we have chosen to use Google Glass as the main head-worn computing device. This decision was made primarily because Glass has the display, touchpad, and IMU sensors and programming Glass is easy (standard Android). But this doesn't exclude other head-worn devices. 

Comments about that our system made the assumption that users have to wear Google Glass first doesn't hold. The work is more about the design exploration of how head orientation can be used for targeting. 

\subsection{Gaze Tracking}
\label{sec:gaze-tracking}
The reason I brought up it here is because gaze tracking seems to be mentioned by multiple people and having a discussion here is good to clear the questions.




%%% Local Variables: 
%%% mode: latex
%%% TeX-master: "uist14"
%%% End: 


\section{Conclusion}
\label{sec:conclusion}

In this project, we explored a novel way of interacting with physical devices by capturing user's attention. The prototype we built demonstrates the possibility of identifying user's line of sight to select and control target appliances. Furthermore, we designed a mechanism to intuitively choose from multiple targets when they are close-by. 

There are still a few more components that require enhancement: (A) Thorough verifications of the IR strength, angles, and reflecting effects. (B) A more robust communication protocol with better error handling. (C) A more stable and richer input method. (D) User study.



%%% Local Variables: 
%%% mode: latex
%%% TeX-master: "main"
%%% End: 


% \section{Acknowledgments}

\bibliographystyle{abbrv}
\bibliography{reference}

\end{document}
