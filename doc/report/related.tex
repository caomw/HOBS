\section{Related Work}
\label{sec:related-work}

In this section, we will summarize some prior work and how it relates to our project in three folds: Tangible User Interface (TUI), Glass-based device, and Attention Aware System (AAS), discussing each in turn. 

\subsection{Tangible User Interface}
\label{sec:tang-user-interf}

The concept of Tangible User Interface was proposed in \cite{Ishii:1997:TBT:258549.258715}, and according to the original paper, such system would enable user to ``grasp \& manipulate bits in the center of users' attention by coupling the bits with everyday physical objects and architectural surfaces''. Thereafter, numerous TUI systems (such as \cite{Nanayakkara:2012:EEF:2212776.2212382, Patel:2006:IPA:2094945.2094962}) are developed. \cite{Merrill:2007:ALP:1758156.1758158} is one that is closest to our system, which base users' interaction with physical objects on people's natural looking, pointing and reaching metaphor. However, these systems have limitations that they focus too much on sensing the physical world, while the actuation is mostly done in cyber world. Recent years, many startups \cite{SmartThings, NinjaBlocks, Lockitron} tries to ``hack'' the physical world to achieve the envisioned Internet of Things by enabling actuating devices from phone/web portal. 

Our system combines some merits from prior work. As has been pointed in Sec.\,\ref{sec:introduction}, we are also using people's natural looking as attention tracking as in \cite{Merrill:2007:ALP:1758156.1758158}, but we are not limited to browsing or searching physical objects. We provide such capability of interacting directly with physical devices. This also improves some of existing work that is based on smartphones or web portal \cite{SmartThings, NinjaBlocks}.

\subsection{Glass Form-factor}
\label{sec:glass-form-factor}

\cite{mann2004continuous}, as the earliest wearable device, uses the form factor of glass to serve as a logging machine that records user's’ daily life. According to one review paper \cite{morris2010emerging}, there are many glass-based device developed for the past decades. Those who use glasses as input have primarily focused on gaze-tracking \cite{Selker:2001:EGE:634067.634176, Nagamatsu:2010:MDG:1753846.1753983}, and the projects that uses glasses for available output device works hard to achieve virtual reality \cite{Lumus, GoogleGlass}.

The most recent and potentially most impactful project \cite{GoogleGlass} integrate many crucial components -- camera, projection, microphone, speaker, IMU and networking -- into a single glass that can achieve virtual reality to an unimaginable extent. We do believe this would become a powerful platform for future interaction development. However, so far the vision is still being limited to digital world. In contrast, our project explores the potential of using glasses for direct interaction with physical devices, rather than performing ``digital'' tasks.

\subsection{Attention Aware System}
\label{sec:attent-aware-syst}

Microsoft Research's project ``Attentional User Interface'' \cite{horvitz2003models} explores how attention can be employed to enhance human-computer interaction, from the lesson learned in human-human interaction. Attention detection can also assist interface design to avoid context shifting overhead, e.g. the peripheral displays and the notification level should be adjusted according to user's attention \cite{parkdesigning}. 

These work researched more on how the system should manage or adapt to users' attention. While for us, we want user to express their attention directly and use this to address the target for interaction.

%%% Local Variables: 
%%% mode: latex
%%% TeX-master: "main"
%%% End: 
