%!TEX root = uist14.tex
\section{Introduction}

%% from the swarm vision to the necessity of selection
The number of smart objects with computation and communication powers have grown rapidly in the past years. Therefore, potential interactive targets in our environment has also increased.
%Such interactions might be {\em explicit} -- controlling smart appliances such as intelligent lighting, AV equipment, or HVAC systems; or {\em implicit} -- tracking the user's attention to gauge interest e.g., for targeted-advertisement or context-aware computing \bjoern{second part needs to be stronger}. 
To initiate such interaction requires {\em selection} information by which a system keeps track of the user's {\em locus of attention}~\cite{raskin} in the world. \sean{this sentence is unclear to me.}

%% previous approaches are limited
Past research has introduced techniques of augmenting hand-held mobile devices with accessories like laser pointers to enable direct aiming at target devices in space~\cite{beigl_point_1999,patel_2-way_2003}. While promising, some drawbacks of using hand-held devices are that the device first has to be retrieved (e.g., from a pocket) and consciously aimed; that the user's hands are required to be free for operation (one to hold the device, one to operate the touch screen or keyboard); and that the user's visual attention is split between looking down at a screen and out at targets in the world. 

%% introducing head-worn computing and head orientation
Emerging head-worn computing devices have the potential to overcome some of these limitations: they do not require retrieval since it is already worn; they may enable hands-free or uni-manual interactions; and they offer near-eye or see-through displays to present information adjacent to physical objects in the wearer's field of view. We thus investigate the research question of how such computing devices may be used for selection of physical targets in the space.

Head-worn devices can naturally exploit the user's head orientation, an important, but also a less precise indicator of the user's locus of attention: it contains the general direction, but not the particular point of focus. While gaze tracking could provide further information about a user's focus, wearable gaze tracking cannot by itself reliably determine the identity of objects in focus (e.g. it also needs to know what the user is seeing). Equipping all devices with stationary gaze trackers is possible~\cite{vertegaal2005media} but much costlier than our approach. We therefore study how to leverage head-orientation alone for target selection in physical spaces. 

The imprecision of human head movement suggests adapting area cursor techniques known from assistive devices~\cite{kabbash1995prince,worden1997making,findlater2010enhanced}. Such techniques employ a two-step selection process: a {\em coarse} selection for an area of interest, followed by a {\em refinement} to select a target within that area.

In this paper, we describe the iterative development and evaluation of an area-selection technique that can be readily implemented with small hardware changes to emerging wearable devices. We augment Google Glass\footnote{\url{http://www.google.com/glass/start/}} to enable infrared (IR) communication between Glass and target appliances. The cone shape of light emitted by an IR LED (a diameter of 60-120cm and distance up to 6m) provides the {\em coarse} selection area. To {\em refine} selection when multiple targets have received IR signals, we first introduce a manual technique where users select a candidate device using a graphical interface on their wearable device.  We then conduct a study with $14$ participants that compares acquisition times for physical targets in a room for our technique and an alternative list selection interface. We find that target acquisition through head orientation is preferred by users and is faster than list selection, given the constraints of linear input using a head-worn touch controller. 

Learning that the refinement UI significantly slowed acquisition times in this study, we designed an improved refinement technique in which target objects compare received IR signal intensity. This intensity value allows the system to automatically selects a target when there is a clear winner. If disambiguation is still required, the fall-back manual UI list is shown with the targets sorted by their intensity values. A second study with $n$ participants shows that IR-intensity disambiguation successfully reduces both the chance of needing to do refinement as well as the time spent in list navigation, comparing to the naive technique from the first study.

One shortcoming discovered in the second study was that participants did not like switching their focuses between the targets in the physical world to the GUI on their near-eye display during the disambiguation process. We intend to solve this problem by introducing the third technique - disambiguate via head motion. The system infers a spatial data structure of target locations using Glass orientation sensors. When disambiguating, Glass senses user head movements and select targets accordingly similar to a joystick controlled by your neck (e.g., nod down to select the target below current selection, tilt right to select next target on the right).
\sean{need a summary of 3rd study here?}

%% three ways
%---
We also demonstrate an example application of our technique used as a remote control of smart appliances: a user looks at the appliance he wishes to control and confirms selection by tapping. An appliance-specific user interface is then shown on the user's near-eye display for further interactions.

%Orientation-based selection enables a wide range of context-aware applications. Examples include smart home remote control, break reminder monitor starer, museum attention tracking, indoor positioning, etc. In Figure\,\ref{fig:teaser}, it's a demonstration of the ``universal remote control'' scenario. The user can easily select the smart appliances by simply looking at it's general direction and confirm such selection with either voice command or by tapping the Glass input pad. Then an appliance-specific control UI will be shown on the head-mounted display. For this application, we have asked 14 participants to try the system and we report the qualitative results from them performing home automation tasks.



%% Contribution
%In summary, this paper makes the following contributions:
%\begin{itemize}
%\item We design and implement a novel head-orientation based selection technique for physical targets based on IR communication. We introduce disambiguation techniques to address the inherent imprecision of head orientation. 
%\item We present evaluations that compare head orientation targeting to list selection and quantify the benefits of automatic disambiguation.
%\item We demonstrate a home appliance remote control application built on top of our selection technique.
%\end{itemize}

%\ben{Having problem fixing the reference}.\bjoern{skip this - we want to be brief.}
%In the remainder of this paper, we will first describe the related works in physical selection and targeting. Given that we focus our scope on head orientation, in Section~\ref{sec:background} we will briefly review human's neck ergonomics as the background for head orientation. In Section~\ref{sec:syst-design-prot}, we then present our system designed for the study of head orientation based selection. The prototype is also what we have used for building the example applications. The disambiguation techniques are discussed in Section~\ref{sec:disamb-techn}, which is followed by the study and evaluation in Section~\ref{sec:evaluation}. To further show the usefulness of having such head orientation-based selection, we describe the enabled applications in Section~\ref{sec:applications}, with detailed implementation about the ``universal remote control'' application. The discussion and conclusion are in Section~\ref{sec:discussion} and Section~\ref{sec:conclusion} respectively. 

%%% Local Variables: 
%%% mode: latex
%%% TeX-master: "uist14"
%%% End: 
