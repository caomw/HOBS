\section{System Design and Prototyping}
\label{sec:syst-design-prot}
Given the background study on using head orientation for targeting, we have designed and prototyped a low-cost system to study how it can be used in a real-world scenario. In this section, we will first briefly describe our design considerations and then for each considerations we propose commercial off-the-shelf (COTS) solutions that enable our prototyping.

\subsection{Design Considerations}
\label{sec:design-cons}

\ben{need to refine this by a better wording. Also we need justification for each!}

{\bf Area selection:} The head muscle's fundamental limitation imposes us to choose an area selection technique. However, there exists such trade-off between the effective size of the beam cursor and the necessity of doing disambiguation. The proper technique should be one that could do both.

% the two visual feedback seems a bit random...
{\bf Leverage visual feedback:} We consider two types of visual feedback. The first type comes from the environment and enables instantaneous visual cue for users to know his head-orientation. The second further helps any type of interaction by embedding rich content information in the near-eye display.
% A calmer~\cite{weiser_coming_1997} approach is to locate visual feedback about selection targets in the environment, to prevent distraction and interruption. Such feedback should be delivered instantaneously, while users look around a room.

{\bf Flexible communication:} Since the system enables interaction with the physical environment, it's important to provide communication between each end-node. With the advance in wireless communication, we can scatter the nodes in an indoor-environment for both testing and deployment.

{\bf Adjustable angle:} We learned from the study about human's head orientation pointing that there is a certain bias and variance per person. To fit each person's line of sight, the system should enable adjustability for calibration.

These design considerations find their expression in our selection of techniques in the following few subsections.

\subsection{IR as beam cursor}
\label{sec:ir-as-beam}

\subsection{Head-up Display}
\label{sec:head-up-display}

\subsection{Wireless Communication}
\label{sec:wirel-comm}

\subsection{3D Printed Adjustable Holder}
\label{sec:3d-print-adjust}




%%% Local Variables: 
%%% mode: latex
%%% TeX-master: "uist14"
%%% End: 
