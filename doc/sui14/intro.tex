%!TEX root = sui14.tex
\vfill
\section{Introduction}
%\bjoern{We should see if we can change language in a few places to strengthen the connection to ``Spatial User Interaction", the title of the conference.}

%% from the swarm vision to the necessity of selection
The number of smart objects in our environment with embedded computation and communication has grown rapidly. These objects are all potential targets for interaction. To initiate {\em spatial interactions}, a user needs to first acquire the target object -- a fundamental task that has been extensively studied in graphical user interfaces, but not yet well-explored in {\em physical spaces}.

Today, companies like Samsung and Whirlpool are making smart appliances with companion applications that use smartphones as {\em universal remote controls}. With these applications, the user can select a device from a list in order to control it with a device-specific user interface. However, this method faces {\em naming} issues (i.e. ``what do we name the lamp on the left?'') and {\em scaling} issues as the number of controlled devices increases. These solutions also present a necessarily flawed mapping from the positions of the appliances in the rich, 3-dimensional world to their place in a 1D or 2D list presented on the screen.
%
%% previous approaches are limited
Past research has used direct aiming at target devices in space with phones to overcome these problems~\cite{beigl_point_1999,patel_2-way_2003}. Such techniques have a few drawbacks: the aiming device first has to be retrieved; the user's hands have to be free for operation; and the user's visual attention is split between looking down at a screen and out at targets in the world. 

%% introducing head-worn computing and head orientation
Emerging head-worn computing devices do not require retrieval since the devices are already worn; they may enable hands-free or uni-manual interactions; and they offer near-eye or see-through displays to present information in the wearer's field of view. We thus investigate how such computing devices may be used for the selection and control of devices in physical spaces. Head-worn devices can naturally exploit the user's head orientation, an important (but imprecise) indicator of the user's {\em locus of attention}~\cite{raskin}. It suggests the general direction, but not the particular point of focus. We draw an analogy to assistive area cursors and adapt area cursor techniques~\cite{kabbash1995prince,worden1997making,findlater2010enhanced} for physical selection. Such techniques employ a two-step selection process: a {\em coarse} selection of an area of interest, followed by a {\em refinement} to select a target within that area.

In this paper, we describe the iterative development and evaluation of \systemnamenospace, an area-selection technique that can be readily implemented with small hardware changes to emerging head-worn devices. We augment Google Glass\footnote{\url{http://www.google.com/glass/start/}} to enable infrared (IR) communication between Glass and target appliances. We contribute and evaluate new methods for addressing selection ambiguity in this context. In all our techniques, the emitted IR beam %(a diameter of 30-60cm and distance up to 8m)%
 provides an initial {\em coarse} selection area (illustrated in Figure~\ref{fig:teaser} {\em left}). To {\em refine} selection when multiple targets have received IR signals, we describe and evaluate three techniques:

 Our {\em Naive IR} technique shows an alphabetically ordered disambiguation list on the near-eye display (Figure~\ref{fig:teaser} center). A study with $14$ participants finds that target acquisition with naive IR targeting is preferred by users and is faster than pure list selection without IR, but refinement is still time consuming.

Our {\em Intensity IR} technique improves refinement as target objects compare IR received signal strength (RSS). This value allows the system to eliminate some peripheral targets and to re-order the refinement interface's list by their intensity values. For example, in Figure~\ref{fig:teaser} of {\em Intensity IR} technique, device 5 is eliminated first and the list is re-ordered based on the intensity readings. A second study with $10$ participants shows that {\em Intensity IR} successfully reduces both the probability of needing to do refinement as well as the time spent in list navigation when compared to {\em Naive IR}.

Our final {\em Head-motion Refinement} addresses the lack of a natural mapping when users select a target in the refinement step using their device's touchpad --- the axes of motion do not map directly to the spatial layout of target devices in a room. We first learn the relative spatial structure of the targets using Glass' orientation sensors. Users can then perform head movements to change selections to spatially adjacent targets (see the right of Figure~\ref{fig:teaser}). For example, nodding down to select the target below current selection, or tilting right to select the next target on the right. We present preliminary feedback from participants on this technique.

We also demonstrate an example application of our technique used as a remote control of smart appliances such as lighting and TV sets: a user looks at the appliance he wishes to control and confirms selection by tapping. An appliance-specific user interface is then shown on the user's near-eye display for further interactions. 

%Orientation-based selection enables a wide range of context-aware applications. Examples include smart home remote control, break reminder monitor starer, museum attention tracking, indoor positioning, etc. In Figure\,\ref{fig:teaser}, it's a demonstration of the ``universal remote control'' scenario. The user can easily select the smart appliances by simply looking at it's general direction and confirm such selection with either voice command or by tapping the Glass input pad. Then an appliance-specific control UI will be shown on the head-mounted display. For this application, we have asked 14 participants to try the system and we report the qualitative results from them performing home automation tasks.


% In summary, this paper makes the following contributions:
% \begin{itemize}
% \item We presented our three iteractions of design.
% \item We present evaluations that compare head orientation targeting to list selection and quantify the benefits of automatic disambiguation.
% \item We demonstrate a home appliance remote control application built on top of our selection technique.
% \end{itemize}



%%% Local Variables: 
%%% mode: latex
%%% TeX-master: "sui14"
%%% End: 
