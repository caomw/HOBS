\section{Discussion}

Study-related topics:
\begin{enumerate}
\item Our technique should have a wider margin as the number of targets increases.

\end{enumerate}

Other topics:
\begin{enumerate}
\item Midas Touch problem? don't want to permanently flash info on screen based on gaze direction.
\item Some fundamental differences to earlier laser-pointer based systems: with a laser pointer in hand one can very precisely point at an arbitrary object and know that one is pointing; however, that alone does not provide feedback on whether a device can actually be controlled or not. By shifting the visual feedback to the target device, 

\item Battery life considerations: we still need to know when to start emitting the IR signal, otherwise, battery life will suffer, which is especially critical for wireless devices. However, requiring complicated steps to turn IR on goes counter the design goals of seamless interactions. A frame-mounted button may be an appropriate design choice.

\item Where to look? For a light, one might put a received on the light itself, or on the light switch. For volume, it might be on a speaker or on the amplifier. Someone should study preferences.
\end{enumerate}

\subsubsection{Discussion}
\sean{to be refined}

\begin{itemize}
\item During the experiment, a lot of wireless messages were transferred for communication and logging purposes. Some messages got lost or delayed and caused the users had to redo that particular action. Infrared mode requires a lot of more handshake communications than the list mode. This means that in a more robust infrastructure, the performance of infrared mode can improve more than that of the list mode.

\item Some of the infrared performance were affected by the strength and stableness of the infrared emitter. By getting a stronger emitter that would be able to reach to a longer distance, connecting to nodes further away such as 4 and 10 would become easier.

\item As shown in figure X, the average time taken to connect to a certain client is correlated to the order of the node within the list. We therefore can assume that as the system scales, the average time required in list mode would have a linear growth. As for the infrared mode, it is more about the density of nodes in a fix space. There’s an upper bound how many smart appliance one can put on a single ta

\item In list mode, users’ eyes focuses more on the display to traverse between the list while in infrared mode, users’ eyes remain in the physical world more.

\item In our prototype, the IR emitter is nearly fixed. Due to different sizes and shapes of the users’ heads, the emitter wouldn’t necessarily point to the center of their eye sights. An adjustable emitter would even further improve the performance of our design.

\item The list mode could also be improved by making the swiping gestures more responsive, e.g. swiping speed affects traversing speed.

\item In our experiment, after users locates the nodes by ID, they can instantly figure out their names by the label provided. In realistic environments, users have to recall the names of the appliances, which would take a lot more time. Furthermore, finding names such as “Living room corner stand lamp” would be more difficult than an alphabet. On the other hand, infrared mode will most likely to be unaffected since it does not require name-recalling or text matching.

 \end{itemize}