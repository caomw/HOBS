%!TEX root = uist14.tex
\section{Discussion}
\label{sec:discussion}
%%%NOTES FROM YANG%%%%
The rapid development of sensing technologies has created many opportunities for new ways to interact with smart objects. In our exploration of the design space of head orientation-based target selection, we carefully selected sensing techniques that are readily available and easy to deploy; and we added complexity to our system only when necessary. 

\subsection{Interpretation of results}

A primary goal of this paper was to provide an effective and efficient method for target selections in physical spaces. Targeting is a fundamental building block across many interaction tasks - it has a significant impact on user experiences collectively and can provide seamless interaction when designed well.

We formalized a scan and refine model of head orientation-based selection (Equation~\ref{eq:time}). We first introduced head orientation as an alternative to list selection and showed that {\em scanning} can outperform list selection. Our redesigns then focused on the case where refinement is needed. The two ways to reduce refinement time are 1) to reduce $P(refine)$, the probability that a manual refinement is necessary; and 2) to reduce $t_{refine}$, the time required to perform the refinement interaction itself. Using IR intensity readings addresses both these terms, as it can be used to both avoid showing refinement dialogs, and to optimize their display when they are needed. 

Our final head-orientation technique improves the nature of the mapping between items in the refinement dialog and the layout of targets in space. Informal testing suggests that users prefer using this spatial mapping.

%Point 1: Why an effective technique for target acquisition is important--clarify the motivation
%This is to address a comment we got from the CHI review. We want to make analogy with target acquisition in the digital world. I copied what we had in the rebuttal here: 
%“Targeting is a fundamental building block across many interaction tasks - it has a significant impact on user experiences collectively and can provide seamless interaction when designed well.”

%Point 2: The overall research methodology
%The rapid development of sensing technologies has created many opportunities for new ways to interact with smart objects. While we explored the design space of head orientation-based target selection, we cautiously selected sensing techniques that are easy to deploy and added complexity to our system only when necessary. Our work thoroughly examined the design space via an iterative design process. Theorize the design space as you had in Equation 1. Ben already had some good points in the draft.

\subsection{Limitations}
We introduced new refinement operations based on IR intensity and spatial adjacency tracking that improve the performance of head orientation in environments with dense targets. These techniques have several limitations. 

First, IR intensity measurements only work within the dynamic range of our sensor. Additional strong IR sources like direct sunlight may saturate the sensor and make discrimination impossible. Second, our adjacency map is built assuming stable relative target locations. If targets move, the map will have to be recalculated. This may be done incrementally during everyday interactions, but we have not yet tackled this challenge. 

Finally, we acknowledge several limitations of our study design: we have not yet systematically studied target density variation, our study was performed in a lab environment, and only measured first use. Future work should study how the technique applies in realistic settings over longer periods of time.
%Point 3: Limitation
%Highlight the novelty of technique 2 & 3 while point out its weakness:
%- How sunlight will impact the techniques
%- Construct the adjacency mapping, what if objects move
%- How robust its adjacency mapping is when the user changes her location or view angle.
%- Outline programming by demonstration opportunities for future research
%- Study limitation: density variation, realistic settings, longitudinal studies to understand how users choose between our techniques and traditional list-based solution. 

\bjoern{removed CV subsection; handled in related work}
%\subsection{Computer-vision Solution?}
%Reviewers suggest considering computer vision-based techniques with markers in the environment (such as QR codes, or Bokode [Mohan] which extends the working distance to a few meters for visual tags). We considered this direction and found several drawbacks, including computational complexity, low accuracy when targets are far, the absence of real-time environment feedback with passive tags, and aesthetic issues with QR codes. While we don't claim our choice of IR is optimal, it has important advantages -- it is low-cost, readily available and quite suitable for area selection. We can weaken our claim and add explanations of how disambiguation techniques might generalize to other methods of implementing target identification.


%%%% PRIOR DRAFT FROM BEN %%%
%\subsection{What is the design space?}
%\label{sec:what-design-space}
%\ben{may change this to a seperate section.}
%Throughout our three iteration, the goal is minimize the overall target selection time (Eq~\ref{eq:time}).

%In our first iteration, by introducing IR to for the {\em scan}, we effectively reduce $P(refine)$ (for a less dense environment) and $t_{refine}$ (reduce the target numbers for the {\em refinement} stage. 

%During the second iteration, with {\em Intensity IR}, $P(refine)$ is the same as {\em Naive IR} while it has been further descreased in a relative dense environment. Also $t_{refine}$ is also reduced because the order of targets in {\em refinement} stage is based on the probability of it being the intended target. 

%Iteraction 3 attempts to solve the problem of mapping targets in the physical space to a 1-d list data structure, and it's mainly in the {\em refinement} stage. Preliminary studies have shown that users prefer this more direct {\em refinement} to the menu-based approach. 

%\ben{Will add a figure to illustrate the above points.}

%\subsection{How dense can the environment be?}
%\label{sec:how-dense-can}
%The motivation of our second iteraction comes from the challenge of a relative dense environment. However, the term {\it dense} is not well-defined in physical space. Given two targets, there are potentially four different parameters to fully describe the {\it density}: separation in x, separation in y, separation in depth and the overall distance from the user to the targets (or the center between them). In this paper, we have focused on the design space of user interaction and a comparison of different techniques. In {\em Intensity IR} technique, we have done a preliminary study of the distribution of IR signal to justify the benefit of using intensity for {\em scan} and  {\em refine} stage. A more systematic and thorough study should be carried to understand the performanc of \systemnamenospace under different scenarios. Though the absolute target acquisition time will change when the environment setup differs, our study results still holds. We leave the {\em density} study as future work.  


%\subsection{Why Google Glass?}
%\label{sec:using-google-glass}
%Google Glass is a well-engineered wearable device that has integrated touchpad, display, and IMU and a couple of wireless radios. This is a suitable platform because we need a head-worn device that can get users' input and display related UI at the near-eye space. The IMU on Google Glass also helps us explore our third technique of monitoring users' head motion for direct {\em refinement}. 

%Before we settled down on Google Glass, we have actually attempted to build a similar glasses form-factor device. Though we successfully integrated the touchpad, and were considering IMU, the near-eye display was difficult to engineer. Among the available commercial off-the-shelf head-up display wearable devices, Google Glass stands out because it's light-weight and designed for everyday use. It's also easy to program (standard Android) with little learning curve.

%\subsection{Computer vision-based physical targeting}
%\label{sec:comp-visi-based}
%One device-free appraoch of performing physical targeting has been brought up during numerous discussions with people when we are working on this project. We see a few challenges in this approach:
%\begin{itemize}
%\item Achieving high accuracy object detection is still an open problem in computer vision. 
%\item problem 2
%\item problem 3
%\end{itemize}

%is to use computer vision to detect \ben{Talk about the challenges in using pure computer vision technique.}

%%% Local Variables: 
%%% mode: latex
%%% TeX-master: "uist14"
%%% End: 
