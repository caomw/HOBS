%!TEX root = uist14.tex
\section{Iteration 3: Orientation-based refinement}
\bjoern{While taking IR intensity into account further reduced the need for manual refinement and increased performance, the UI navigation scheme is problematic - it is not spatially related in any meaningful way to the layout of targets in a room. A better interaction technique would respect that ordering. For example, when refinement is needed, tilting the head slightly to the right could select the target that's adjacent to the right of the currently selected target.

Such spatial navigation requires knowledge about the layout of targets in the environment. However, one of the strengths of our technique so far is that it does not require any map ahead of time. To enable some spatial navigation, we introduce a final iteration in which we build up a spatial data structure by demonstration (i.e., the user looks around the room) and then leverage that data structure during the refinement step of our interaction.}

\bjoern{This technique is based on the assumption that users will generally select targets in indoor environments where targets are spread around the periphery. These assumptions enable us to use orientation data without knowing the user's absolute position.}

\subsection{Implementation}
\bjoern{give implementation details}

\subsection{Informal User Feedback}
\bjoern{We informally evaluate this technique with N users: ...}