\section{Introduction}
\label{sec:introduction}

There has been the vision of smart-home and smart-office where everything around is intelligent enough to assist human in everyday life. The development of radio technology, sensors, and actuators are making this dream come true. You can buy those basic components online \cite{SmartHome, NinjaBlocks} and many hobbyists have already created their own smart environment (see \cite{BRAD} as an example). Recently, with the the prevalence of smartphones, many systems \cite{SmartThings, Lockitron} have been integrated into smartphone applications to lower the barrier for ordinary users.

Admittedly, people have been used to accessing the Internet, conversing with friends, and even dealing with online business through their smartphones. These handheld devices serve as an important portal for users to be connected to the Internet. And now many systems are making smartphones the portal to interact with physicalworld. Given all the benefits we can get by using smartphones -- networking,  display, gesture input, and increasing computability,  we argue that traditional GUI on mobile platform still imposes difficulty in interacting with physical objects. Take iPhone for example, users are required to browse a list of possible icons and then select the right one. Even on Android platform, you can create widget to simplify the browsing overhead. We hold the opinion that squeezing physically distributed objects onto a-few-inch screen is not the right approach. And a more direct approach is desirable.

In this project, we aim to explore the ability of direct interaction by using glass-based attention tracking to pin-point the target. 

% explain why we need glass form factor


The rest of this paper is structured as follows. In Sec.\,\ref{sec:related-work}, we will give an overview of some related projects. Then we present our approach in detail about the system design and implementation in Sec.\,\ref{sec:design} and Sec.\,\ref{sec:implementation} respectively. Then Sec.\,\ref{sec:evaluation} shows our evaluation results and we provide an in-depth discussion (Sec.\,\ref{sec:discussion}) about different kinds of interaction since not all are suitable for our system. Sec.\,\ref{sec:conclusion} concludes this paper.


%%% Local Variables: 
%%% mode: latex
%%% TeX-master: "main"
%%% End: 
